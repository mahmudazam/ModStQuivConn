\documentclass[11pt]{amsart}

\usepackage{amsmath}
\usepackage{amsthm}
\usepackage{amsfonts}
\usepackage{amssymb}
\usepackage{amsthm}
\usepackage{bbm}
\usepackage{enumitem}
  \setlist{itemsep=1em}
  \setenumerate{label=(\roman*)}
  \setitemize{leftmargin=1.5em}
\usepackage[mathscr]{euscript}
\usepackage{float}
\usepackage{ifthen}
\usepackage{listings}
\usepackage{mathdots}
\usepackage{mleftright}
\usepackage[new]{old-arrows}
\usepackage{rotating}
\usepackage{sansmath}
\usepackage{tabularx}
\usepackage{tikz}
  \usetikzlibrary{positioning}
\usepackage{tikz-cd}
  \tikzcdset{
    cells={font=\everymath\expandafter{\the\everymath\displaystyle}},
  }
\usepackage{todonotes}
\usepackage{xcolor}
\usepackage{xspace}

\usepackage[pdftex,pdfpagelabels,bookmarks,hyperindex,hyperfigures]{hyperref}
  \hypersetup{
    colorlinks,
    linkcolor={red!55!black},
    urlcolor={blue!55!black},
    citecolor={green!55!black}
  }
\usepackage{cleveref}

\definecolor{codegreen}{rgb}{0,0.6,0}
\definecolor{codegray}{rgb}{0.5,0.5,0.5}
\definecolor{codepurple}{rgb}{0.58,0,0.82}
\definecolor{backcolour}{rgb}{0.95,0.95,0.92}
\lstdefinestyle{mystyle}{
    xleftmargin={2.5em},
    %xrightmargin={2.5em},
    backgroundcolor=\color{backcolour},
    commentstyle=\color{codegreen},
    keywordstyle=\color{magenta},
    numberstyle=\tiny\color{codegray},
    stringstyle=\color{codepurple},
    basicstyle=\ttfamily\footnotesize,
    breakatwhitespace=false,
    breaklines=true,
    captionpos=b,
    keepspaces=true,
    numbers=left,
    numbersep=5pt,
    showspaces=false,
    showstringspaces=false,
    showtabs=false,
    tabsize=2
}
\lstset{style=mystyle}


\usepackage{subfiles}

\newtheorem{thm}{Theorem}
\numberwithin{thm}{section}

\newtheorem{cthm}[thm]{Theorem (under a Conjecture)}
\newtheorem{lem}[thm]{Lemma}
\newtheorem{prop}[thm]{Proposition}
\newtheorem{cnj}[thm]{Conjecture}
\newtheorem{guess}[thm]{Guess}
\newtheorem{cor}[thm]{Corollary}
\newtheorem{ccor}[thm]{Corollary (under a Conjecture)}

\theoremstyle{definition}

\newtheorem{cns}[thm]{Construction}
\newtheorem{conv}[thm]{Convention}
\newtheorem{exm}[thm]{Example}
\newtheorem{defn}[thm]{Definition}
\newtheorem{fact}[thm]{Fact}
\newtheorem{notn}[thm]{Notation}
\newtheorem{rmk}[thm]{Remark}
\newtheorem{warn}[thm]{Warning}
\newtheorem{qsn}[thm]{Question}
\newtheorem{asmptn}[thm]{Assumption}

\newcommand{\bb}{\mathbb}
\newcommand{\bbm}{\mathbbm}
\newcommand{\f}{\mathfrak}
\newcommand{\s}{\mathscr}
\newcommand{\p}{\mathsf}
\newcommand{\pb}{\mathsfbf}
\newcommand{\mc}{\mathcal}

\newcommand{\bA}{\bb{A}}
\newcommand{\bC}{\bb{C}}
\newcommand{\bP}{\bb{P}}
\newcommand{\bR}{\bb{R}}
\newcommand{\bk}{\bbm{k}}
\newcommand{\one}{\bbm{1}}
\newcommand{\bZ}{\bb{Z}}
\newcommand{\bQ}{\bb{Q}}
\newcommand{\sK}{\s{K}}
\newcommand{\sA}{\s{A}}
\newcommand{\sB}{\s{B}}
\newcommand{\sC}{\s{C}}
\newcommand{\sD}{\s{D}}
\newcommand{\sE}{\s{E}}
\newcommand{\sF}{\s{F}}
\newcommand{\sP}{\s{P}}
\newcommand{\cP}{\mc{P}}
\newcommand{\sS}{\s{S}}
\newcommand{\mcX}{\mc{X}}

\newcommand{\wh}{\widehat}
\newcommand{\br}[1]{\mleft( #1 \mright)}
\newcommand{\sbr}[1]{\mleft[ #1 \mright]}
\newcommand{\gen}[1]{\mleft\langle #1 \mright\rangle}
\newcommand{\set}[2][]{
  \ifthenelse{\equal{#1}{}}{
    \mleft\{ #2 \mright\}
  }{
    \mleft\{ #1\ :\ #2 \mright\}
  }
}

\renewcommand{\to}[1][]{
  \ifthenelse{\equal{#1}{}}{
    \longrightarrow
  }{
    \stackrel{#1}{\longrightarrow}
  }
}

\newcommand{\To}[1][]{
  \ifthenelse{\equal{#1}{}}{
    \Longrightarrow
  }{
    \stackrel{#1}{\Longrightarrow}
  }
}

\renewcommand{\mapsto}[1][]{
    \ifthenelse{\equal{#1}{}}{
      \longmapsto
    }{
      \stackrel{#1}{\longmapsto}
    }
}

\newcommand{\ot}[1][]{
  \ifthenelse{\equal{#1}{}}{
  \longleftarrow
  }{
    \stackrel{#1}{\longleftarrow}
  }
}

\newcommand{\adj}{\text{ad}\ }
\renewcommand{\ker}[1]{\text{ker}\br{#1}}
\newcommand{\im}{\mathrm{im}\xspace}
\newcommand{\gl}[1]{\pb{gl}\br{#1}}
\newcommand{\Gl}{\mathrm{Gl}}
\newcommand{\Mat}{\mathrm{Mat}}
\newcommand{\SL}[1]{\pb{sl}\br{#1}}
\newcommand{\bmat}[1]{\begin{bmatrix} #1 \end{bmatrix}}
\newcommand{\spn}[1]{\text{span}\set{#1}}
\newcommand{\tr}{\text{tr}\xspace}
\newcommand{\End}{\mathrm{End}}
\newcommand{\Spn}{\mathrm{Span}}
\newcommand{\lrarrow}{\stackrel{\rightarrow}{\leftarrow}}
\newcommand{\hto}[1][]{\stackrel{#1}{\longhookrightarrow}}
\newcommand{\lto}[1][]{\stackrel{#1}{\longrightarrow}}
\newcommand{\eps}{\varepsilon}
\newcommand{\del}{\delta}
\newcommand{\id}{\mathrm{id}}
\newcommand{\ev}{\mathrm{ev}}

\renewcommand{\Ref}[2][]{\ifthenelse{\equal{#1}{}}{\ref{#2}}
                      {\hyperref[#2]{\ref*{#1}(\ref*{#2})}}}
\newcommand{\Aref}[2][]{\ifthenelse{\equal{#1}{}}{\autoref{#2}}
                      {\hyperref[#2]{\autoref*{#1}\ref*{#2}}}}
\newcommand{\Sref}[1]{\hyperref[#1]{\S \ref*{#1}}}
\newcommand{\dom}{\mathrm{dom}}
\newcommand{\codom}{\mathrm{codom}}

\newcommand{\Field}{\mathrm{Field}}
\newcommand{\Set}{\mathrm{Set}}
\newcommand{\Fr}[1]{\mathrm{Frac}\br{#1}}
\newcommand{\Gal}{\mathrm{Gal}}
\newcommand{\Grp}{\mathrm{Grp}}
\newcommand{\Grpd}{\mathrm{Grpd}}
\newcommand{\op}{\mathrm{op}}
\newcommand{\Cov}{\mathrm{Cov}}
\newcommand{\Et}{\mathrm{Et}}
\newcommand{\Sub}{\mathrm{Sub}}
\newcommand{\Deck}{\mathrm{Deck}}
\newcommand{\mono}{\mathrm{mono}}
\newcommand{\Etale}{\mathrm{Etale}}
\newcommand{\PSh}{\mathrm{PSh}}
\newcommand{\PShi}{\mathrm{PSh}_\infty}
\newcommand{\Sh}{\mathrm{Sh}}
\newcommand{\Fun}{\mathrm{Fun}}
\newcommand{\Top}{\mathrm{Top}}
\newcommand{\Pnt}{\mathrm{Pnt}}
\newcommand{\Aut}{\mathrm{Aut}}
\newcommand{\Sing}{\mathrm{Sing}}
\newcommand{\Nrv}{\mathrm{N}}
\newcommand{\Shi}{\Sh^\infty_{\sS}}
\newcommand{\Pro}{\mathrm{Pro}}
\newcommand{\Hom}{\mathrm{Hom}}
\newcommand{\HHom}{\mathcal{H}\mathrm{om}}
\newcommand{\Map}{\mathrm{Map}}
\DeclareMathOperator{\holim}{\mathrm{holim}}
\DeclareMathOperator{\hocolim}{\mathrm{hocolim}}
\DeclareMathOperator{\colim}{\mathrm{colim}}
\newcommand{\Tors}{\mathrm{Tors}}
\newcommand{\Topoi}{\s{T}op_\infty}
\newcommand{\Sc}[1][]{\ifthenelse{\equal{#1}{}}{\mathrm{Sch}}{\mathrm{Sch}_{/#1}}}
\newcommand{\dSc}[1][]{\mathrm{d}\Sc[#1]}
\newcommand{\ASc}{\mathrm{AffSch}}
\newcommand{\Spec}{\mathrm{Spec}}
\newcommand{\Spf}{\mathrm{Spf}}
\newcommand{\Proj}{\mathrm{Proj\ }}
\newcommand{\CP}{\mathbb{C}P}
\newcommand{\Quot}{\mathrm{Quot}}
\newcommand{\icat}{$\infty$--category\xspace}
\newcommand{\Cat}{\s{C}\mathrm{at}}
\newcommand{\Catinf}{\Cat_\infty}
\newcommand{\sCat}{\mathrm{s}\Cat}
\newcommand{\sSet}{\mathrm{sSet}}
\newcommand{\Mot}{\mathrm{Mot}}
\newcommand{\QCoh}{\mathrm{QCoh}}
\newcommand{\VecBun}{\mathrm{VecBun}}
\newcommand{\Vect}{\mathrm{Vect}}
\newcommand{\Higgs}{\mathrm{Higgs}}
\newcommand{\Conn}{\mathrm{Conn}}
\newcommand{\Hodge}{\mathrm{Hodge}}
\newcommand{\Perf}{\mathrm{Perf}}
\newcommand{\Bun}{\mathrm{Bun}}
\newcommand{\Sym}{\mathrm{Sym}\xspace}
\newcommand{\St}{\mathrm{St}}
\newcommand{\AlgSt}{\mathrm{AlgSt}}
\newcommand{\Sch}{\mathrm{Sch}}
\newcommand{\FSch}{\mathbb{F}\mathrm{Sch}}
\newcommand{\DSt}{\mathrm{DSt}}
\newcommand{\SpSt}{\mathrm{SpSt}}

\newcommand{\et}{\text{\'et}}
\newcommand{\fppf}{\text{fppf}}

\newcommand{\Mod}{\mathrm{Mod}}
\newcommand{\LMod}{\mathrm{LMod}}
\newcommand{\CRing}{\mathrm{CRing}}
\newcommand{\CAlg}{\mathrm{CAlg}}
\newcommand{\Mon}{\mathrm{Mon}}
\newcommand{\Zar}{\mathrm{Zar}}
\newcommand{\Nis}{\mathrm{Nis}}
\newcommand{\Gr}{\mathrm{Gr}}
\newcommand{\Ab}{\mathrm{Ab}}
\newcommand{\Ob}[1]{\mathrm{Ob}\br{#1}}
\newcommand{\Tr}{\mathrm{tr}}
\newcommand{\Kan}{\s{K}\mathrm{an}}
\newcommand{\Aff}{\mathrm{Aff}}
\newcommand{\ASt}{\mathrm{AlgSt}}
\newcommand{\str}{\mathrm{str}}
\newcommand{\unstr}{\mathrm{unstr}}
\newcommand{\stb}{\mathrm{stb}}


\usepackage[style=alphabetic]{biblatex}
\renewcommand{\subtitlepunct}{: }
\addbibresource{refs.bib}
\AtNextBibliography{\small}

\title[Quiver Bundles and Non-Abelian Cohomology]
      {Quiver Bundles and Non-Abelian Cohomology}

\author{Mahmud Azam}
\email{mahmud.azam@usask.ca}

\author{Steven Rayan}
\email{rayan@math.usask.ca}

\address{Centre for Quantum Topology and Its Applications (quanTA) and
Department of Mathematics and Statistics, University of Saskatchewan, SK,
Canada~ S7N 5E6}

\begin{document}

\begin{abstract}
\end{abstract}

\maketitle

\tableofcontents

%==============================================================================%

\section{Introduction}

The main aspect of non-Abelian Hodge theory
is a homeomorphism between the moduli spaces of polystable Higgs bundles
with vanishing rational Chern classes and semisimiple flat connections over
a smooth projective variety $X$. We may wonder:
if ``continuous'' families of Higgs bundles are equivalent to those of flat
connections, then families of morphisms of Higgs bundles should have some close
relation to families of morphisms of flat connections.
As a first step towards formalizing this idea, in \cite{ModQuivBun}
in \cite{ModQuivBun}, we constructed a moduli stack parametrizing tuples
$(E, \phi, F, \psi, f)$, where $E, F$ are vector bundles on $X$,
$\phi, \psi$ are Higgs fields on $E, F$ respectively, and $f : E \to F$
is a morphism of Higgs bundles. We showed that with fairly mild constraints
on an algebraic stack $X$ over an algebraically closed field $k$, which includes
all projective varieties, this moduli stack is algebraic.
Denote this moduli stack as $\Higgs_1(X)$. A natural next step is
to construct an algebraic stack parametrizing tuples $(E, \phi, F, \psi, f)$,
where $E, F$ are again vector bundles on $X$, $\phi, \psi$ are flat connections
on $E, F$ respectively, and $f : E \to F$ is morphism of connections.
In the present work, we construct this stack --- call it $\Conn_1(X)$ ---
along with a stack, denoted $\Hodge_1(X)$ over the affine line whose fibres over
non-zero points are copies of $\Vect_1(X_{dR})$ and the fibre over $0$ is
$\Higgs_1(X)$. Of course, taking the fibre over $0$ gives an alternate
construction of $\Higgs_1(X)$. We show that $\Conn_1(X)$ is algebraic
which implies that $\Conn_1(X)$ is algebraic. Again, this gives an alternate
proof of the algebraicity of $\Higgs_1(X)$.

\subsection{Notation and Conventions}

We fix a base scheme $S$ and consider stacks on the category $\Sch_{/S}$ of
schemes over $S$ with respect to the $\fppf$ topology.
We denote the strict $2$--category of such objects as $\St_{/S}$.
We denote the category of formal schemes over $S$ as $\FSch_{/S}$, which
of course embeds fully faithfully into $\St_{/S}$.

As we have already mentioned in the introduction, we use the following
notation for the moduli stacks (left) of the various parametrized
objects (right):
\begin{center}
\begin{tabular}{c l}
$\Vect$ & vector bundles over $S$ \\
$\Vect(\mcX/S)$ & vector bundles over $\mcX$ \\
$\Vect_1(\mcX/S)$ & vector bundle morphisms over $\mcX$ \\
$\Higgs(\mcX/S)$ & Higgs bundles over $\mcX$ \\
$\Higgs_1(\mcX/S)$ & Higgs bundle morphisms over $\mcX$ \\
$\Conn(\mcX/S)$ & vector bundles with connections over $\mcX$ \\
$\Conn_1(\mcX/S)$ & morphisms of connections over $\mcX$
\end{tabular}
\end{center}
Note that this is a slight departure from \cite{ModQuivBun}. We follow this
convention, first, to be consistent with the notation $\Perf(X/S)$
for the moduli stack of perfect complexes, as used in \cite{GeomNonAbHodgeFilt},
and, second, because the $\mc{M}$ in $\mc{M}_{\Vect(X)}$, for example,
seems redundant.

%==============================================================================%

\section{Quotient Stacks of Formal Groupoids}

In this subsection we review formal groupoids and the stacks associated
to them as described in \cite[\S 3.1]{GeomNonAbHodgeFilt}. These are central
to the stack-based formulation of non-Abelian Hodge theory. Specifically,
the stacks of Higgs bundles and connections and deformations thereof
are all defined as mapping stacks out of stacks associated to formal
groupoids.

\subsection{Definitions}

\begin{defn}[Formal Groupoid]
\label{defn:form-cat}
A formal category over $S$ is an internal category $(X, \sF, s, t, c, i)$ in
$\FSch_{/S}$ such that
\begin{enumerate}
\item $X$ is a scheme in $\Sch_{/S}$,
\item $i : X \to \sF$ is a closed immersion realizing $X$ as the underlying
scheme of $\sF$.
\end{enumerate}
This data is called a formal groupoid if, in addition:
\begin{enumerate}[resume]
\item for each $U \in \Sch_{/S}$, $(X(U), \sF(U), s_U, t_U, c_U, i_U)$
is a groupoid (internal to $\Set$).
\end{enumerate}
A formal category as above is said to be smooth if
\begin{enumerate}[resume]
\item the structure map $X \to S$ is smooth,
\item the morphisms $s, t : \sF \to X$ are formally smooth.
\end{enumerate}
\end{defn}

\begin{notn}
We will write a formal category as above simply as $(X, \sF)$,
when the structure maps are clear from context.
\end{notn}

For convenience, we make the following definition:
\begin{defn}[Formal Stack]
\label{defn:form-st}
A formal stack is a stack $\mathcal{X} \in \St_{/S}$ such that there exists
a formal groupoid $(X, \sF)$ and a $2$--coequalizer diagram in $\St_{/S}$:
\[\begin{tikzcd}
\sF \ar[r, shift left, "s"] \ar[r, shift right, "t" below] &
X \ar[r] &
\mathcal{X}
\end{tikzcd}\]
presenting $\mathcal{X}$ as a quotient stack.
We call $\mathcal{X}$ the stack associated to the formal groupoid
$(X, \sF)$.
\end{defn}

\begin{warn}
A formal stack is not necessarily a formal algebraic stack as defined in
\cite[Definition 5.3]{FormalAlgSt}. For instance, it may have a digonal
not representable by algebraic spaces --- see \cref{prop:Hodge-st-diag-non-rep}
--- which contradicts \cite[Lemma 5.12.]{FormalAlgSt}.
On the other hand, a formal algebraic stack is not necessarily a formal stack
since it may be a quotient of formal algebraic spaces more general than formal
schemes \cite[47]{FormalAlgSt}.
\end{warn}

\begin{notn}
In the context of the above definition, if the formal groupoid $(X, \sF)$
is clear from context, then we write $\mathcal{X}$ as $X_\sF$.
\end{notn}

\subsection{Main Examples}

There are three main examples of interest to us \cite[31---33]{NonAbHodgeFilt}.
The rough idea behind all of these is that a quasicoherent sheaf on
a formal stack is $(X, \sF)$ is a quasicoherent sheaf on $X$ along with
isomorphisms between the stalks encoded by $\sF$. By varying $\sF$,
and consequently the isomorphisms of stalks, we can recover connections and
Higgs bundles.

\begin{exm}[de Rham Stack]
\label{exm:dR-st}
If $X \to S$ is separated, then the diagonal $\Delta_{X/S} : X \to X \times_S X$
is a closed immersion and we take $\sF \to[{(s, t)}] X \times_S X$ to be the
formal completion of $X \times_S X$ along the set theoretic image
$\Delta_{X/S}(X)$. The composition morphisms $c : \sF \times_X \sF \to \sF$
is the one induced by the map 
\todo{Make this more precise.}
$(X \times_S X) \times_X (X \times_S X) \to X$.
The identity morphism $i : X \to \sF$
is simply the closed immersion into the formal completion.
We denote $\sF$ by $\sF_{dR}$ and the associated stack over $S$, by
$X_{dR} \to S$, in this case.
\end{exm}

\begin{exm}[Dolbeault Stack]
\label{exm:Dol-st}
If $X \to S$ is again separated, then the diagonal
$\Delta_{X/S} : X \to X \times_S X$ is again a closed immersion. Furthermore,
any section of a separated morphism is a closed immersions. Since the
projection $T(X \times_S X) \to X \times_S X$ of the tangent bundle of
$X \times_S X$ is affine and hence separated, the zero section
$0_X : X \to T(X \times_S X)$ is a closed immersion. Therefore, the composite
\[
\Delta' : X \to[\Delta_{X/S}] X \times_S X \to[0_X] T(X \times_S X)
\]
is a closed immersion.
We can then take $\sF \to T(X \times_S X)$ to be the
formal completion of $T(X \times_S X)$ along the set theoretic image
$\Delta'(X)$.
Then, the map $\sF \to[{(s, t)}] X \times_S X$ is obtained by
composing with the bundle projection.
The composition morphism $c : \sF \times_X \sF \to \sF$ is induced by
the addition morphism \todo{Make this more precise.}
\[
+ : T(X \times_S X) \times_{X \times_S X} T(X \times_S X) \to T(X \times_S X)
\]
The identity morphism $i : X \to \sF$
is again the closed immersion into the formal completion.
We denote $\sF$ by $\sF_{Dol}$ and the associated stack over $S$ by
$X_{Dol} \to S$, in this case.
\end{exm}

\begin{exm}[Hodge Stack]
\label{exm:Hodge-st}
Take a separated morphism of schemes $X \to T$ over $\bZ$,
consider a formal groupoid over $S := \bA^1_T = \bA^1_\bZ \times_\bZ T$
whose scheme of objects
is $X \times_T \bA^1_T \to \bA^1_T$. The formal scheme of morphisms is
defined as follows. We take the blow up $B \to X \times_T X \times_T \bA^1_T$ of
$(X \times_T \bA^1_T) \times_{\bA^1_T} (X \times_T \bA^1_T)
\cong (X \times_T X \times_T \bA^1_T)$ along the set theoretic image of the
map
$\Delta_{X/T} \times_T 0 : X \cong X \times_T T
\to X \times_T X \times_T \bA^1_T$ where
$0 : T \to \bA^1_T$ is the pullback of the map
$\Spec(\bZ[x] \to \bZ : x \mapsto 0)$ along the structure map
$T \to \Spec(\bZ)$,
and the blow up $B' \to X \times_T X$ of $X \times_T X$ along the image of
$\Delta_{X/T}$.
These fit into a commutative square:
\[\begin{tikzcd}
B' \ar[r] \ar[d] &
B \ar[d] \\
X \times_T X  \ar[r, "{(\id, 0)}"] & X \times_T X \times \bA^1_T
\end{tikzcd}\]
where the top and bottom arrows are both closed embeddings.
\todo{Provide reference --- Annala's notes: an option}
Then, $B'$ is the strict transform of $\im(\id_{X \times_T X}, 0)$
in the blow up of $X \times_T X \times_s \bA^1_T$ along
$\im(\Delta_{X/T} \times_T 0)$ by
\cite[\href{https://stacks.math.columbia.edu/tag/080E}{Lemma 080E}]
{stacks-project}.
We take $Y$ to be the complement of $B'$ in $B$ --- note that
$Y$ is an open subscheme of $B$ since the image of $B'$ is closed.
We choose a closed embedding $\Delta' : X \times_T \bA^1_T \to Y$ making
the following diagram commute:
\[\begin{tikzcd}[column sep=huge]
& Y \ar[d] \\
X \cong X \times_T T
  \ar[r, "\Delta_{X/T} \times_T 0" below]
  \ar[ru, "\Delta'" above left] &
X \times_T X \times_T \bA^1_T
\end{tikzcd}\]
Then, the formal scheme of morphisms is taken to be the formal completion
$\sF \to Y$ of $Y$ along $\Delta'(X)$. The map
$\sF \to[{(s, t)}] X \times_T X \times_T \bA^1_T$ is given by the two
projections $Y \to X \times_T X \times_T \bA^1_T \to X \times_T \bA^1_T$.
The composition map $c : \sF \times_{X \times_T \bA^1_T} \sF \to \sF$
is induced by the map:
\[
(X \times_T X \times_T \bA^1_T) \times_{X \times_T \bA^1_T}
(X \times_T X \times_T \bA^1_T) \to (X \times_T X \times_T \bA^1_T)
\]
The identity morphism $i : X \times_T \bA^1 \to \sF$
is again the closed immersion into the formal completion.
We will write $\sF$ as $\sF_{Hod}$ and the associated stack over $S = \bA^1_T$
as $X_{Hod} \to \bA^1_T$, in this case.
\end{exm}

These stacks are related to each other by the following well known fact which
is the main connection with non-Abelian Hodge theory.

\begin{prop}
Let $X \to T$ be a separated morphism of schemes and consider
$X_{dR}$ and $X_{Dol}$ by taking $S = T$ in \cref{exm:dR-st} and
\cref{exm:Dol-st}, respectively.
Also consider $X_{Hod}$ by taking $S = \bA^1_T$ in
\cref{exm:Hodge-st}. Then, for any closed point $\lambda : T \to \bA^1_T$,
the fibre $X_{Hod, \lambda}$ of $X_{Hod} \to \bA^{1}_T$ over $\lambda$ is
equivalent as a stack to $X_{dR}$ when $\lambda \neq 0$, and is equivalent to
$X_{Dol}$, when $\lambda = 0$.
\end{prop}
\begin{proof}
\todo[inline]{Complete the proof}
\end{proof}

We also record here some basic facts about these stacks that make it difficult
to apply some of the techniques of \cite{ModQuivBun} directly ---
see \cref{sec:quiv-bun-form-st}.

\begin{cor}\label{prop:dR-st-set-fibres}
The stack $X_{dR}$ of \cref{exm:dR-st} is a stack whose fibre categories are
setoids (a groupoid that is also a preorder, and hence a groupoid with
contractible connected components). That is, it is equivalent to the
Grothendieck construction of a sheaf of sets.
\end{cor}
\begin{proof}
This follows from the fact that the $\sF_{dR} \to X \times_S X$ is monomorphism
of sheaves of sets.
\end{proof}

\begin{prop}\label{prop:Hodge-st-diag-non-rep}
Taking $T = \Spec(k)$ for some algebraically closed field $k$ in the
context of \cref{exm:Hodge-st},
$X_{Hod}$ does not have a diagonal representable
by algebraic spaces.
\end{prop}
\begin{proof}
We consider a $T= \Spec(k)$--point
$x \in X_{Hod}(\Spec(k))$ that factors as a map
$\Spec(k) \to[{(x', 0)}] X \times_k \bA^1_k \to X_{Hod}$,
and its stabilizer $\stb(x)$. Recalling that the preimage of a point in a
blow-up is a projective space, we have that the stabilizer
fits into the following pasting of Cartesian squares:
\[\begin{tikzcd}
\stb(x) \ar[r] \ar[d] \ar[rd, phantom, "\lrcorner" very near start] &
\bP^n_k \setminus \bP^{n - 1}_k \ar[r] \ar[d]
  \ar[rd, phantom, "\lrcorner" very near start]&
\Spec(k) \ar[d, "{(x', x', 0)}"] \\
\sF_{Hod} \ar[r] & Y \ar[r] & X \times_k X \times_k \bA^1_k
\end{tikzcd}\]
Now, $\sF_{Hod}$ is the formal completion of $Y$ along $\Delta'(X)$, while
the preimage of $\Delta'(X)$ in $\bP^n_k \setminus \bP^{n - 1}_k$
is a single point $x''$ lying over $(x', x', 0)$. By the compatibility of
formal completions with fibre products
\cite[\href{https://stacks.math.columbia.edu/tag/0APV}{Lemma 0APV}]
{stacks-project}, we have that $\stb(x)$ is the formal completion
of $\bP^{n}_k \setminus \bP^{n - 1}_k$ along a point, which is just
the formal completion of an affine chart containing that point along that
point. That is, up to change of coordinates, we have:
\[
\stb(x) = \Spf(k[[x_0, \dots, x_n]])
\]
where $\Spf$ denotes the formal spectrum functor. If $\stb(x)$ were
representable by an algebraic space, then it would have to be a scheme
by \cite[Corollary 3.1.2]{NagataComp} as the reduction $\stb(x)_{red}$ is
$\Spec(k)$. $\stb(x)$ would further have to be an affine scheme by
\cite[\href{https://stacks.math.columbia.edu/tag/06AD}{Lemma 06AD}]
{stacks-project}.
However, it is easy to see that
$\Spf(k[[x_0, \dots, x_n]])$ is not an affine scheme: as a locally
ringed space, it consists of a single point whose stalk is
$k[[x_0, \dots, x_n]]$. For any ring $A$ where $|\Spec(A)|$ is a point,
the stalk at that point must be $A$. If $\stb(x) = \Spec(A)$,
then we must have $A = k[[x_0, \dots, x_n]]$, but the latter is a DVR
and hence has two prime ideals: namely $(0)$ and $(x_0, \dots, x_n)$. This is a
contradiction. Thus, $\stb(x)$ cannot be an algebraic space
and $X_{Hod}$ cannot have diagonal representable by algebraic spaces.
\end{proof}

\begin{prop}\label{prop:Dol-st-diag-non-rep}
In the context of \cref{exm:Dol-st}, taking $S = \Spec(k)$ for some
algebraically closed field $k$ and $X \to \Spec(k)$ to be smooth,
the stack $X_{Dol}$ does not have a diagonal representable by algebraic spaces.
\end{prop}
\begin{proof}
Consider a point $x \in X_{Dol}(\Spec(k))$ that factors as
$\Spec(k) \to[x'] X \to X_{Dol}$.
\todo[inline]{Complete the proof}
\end{proof}

\subsection{Vector Bundles on Formal Stacks}

\todo[inline]{Maybe review the relation with $\Lambda$--modules here
\cite[30]{NonAbHodgeFilt}}

The following should be well known but we present it here nevertheless
for completeness.

\begin{thm}\label{thm:Vect-proj-form-st}
Let $(X, \sF)$ be a smooth formal stack over an algebraically closed field
$k$ of characteristic $0$ such that the structure maps
$X \Spec(k)$ are projective morphisms. Then,
$\Vect(X_\sF/S)$ is algebraic and locally of finite presentation.
\end{thm}
\begin{proof}
We know that:
\[
\Vect(X_\sF/S) = \coprod_{n = 0}^\infty \Map(X_\sF, B\Gl_n(k))
\]
Then, it is known that $\Map(X_\sF, B\Gl_n(k))$ is algebraic and locally of
finite type --- see the proof of \cite[Theorem 7.2]{AlgGeom-n-St}.
Then, $\Vect(X_\sF/S)$ is also an algebraic stack locally of finite
presentation.
\todo{Need to check whether ``projective formal groupoid'' really means
what we described above.}
\end{proof}

%==============================================================================%

\section{Arrow Bundles}

In this section we review the moduli stack $\Vect_1(\mcX/S)$ of vector bundle
morphisms (or triples) as defined in \cite[\S 3]{ModQuivBun}.
This stack parametrizes triples $(E, F, f)$ consisting of two vector bundles
$E, F$ and a morphism of vector bundles $f : E \to F$ over a fixed base stack
$\mcX \in \St_{/S}$. The main purpose of this section is to give a concrete
description of the fibre categories $\Vect_1(\mcX/S)(U)$ over an scheme
$U \in \Sch_{/S}$ and the boundary map
\[
\Vect_1(\mcX/S) \to[{(s, t)}] \Vect(\mcX/S) \times \Vect(\mcX/S)
\]
both of which we mentioned in \cite{ModQuivBun} without precise descriptions
or proofs.

\subsection{The Moduli Stacks}

We first recall the definition of $\Vect_1(\mcX/S)$ and the face maps
$s, t : \Vect_1(\mcX/S) \to \Vect(\mcX/S)$.

\todo[inline]{Recall definitions here}

\subsection{Concrete Descriptions}

\begin{thm}\label{thm:Vect1-psfunctor}
For any stack $\mcX \in \St_{/S}$ and any scheme $U \in \Sch_{/S}$,
the $2$--sheaf of groupoids associated to $\Vect_1(\mcX/S)$ via the Grothendieck
construction is defined as follows:
\begin{itemize}
\item For any $U \in \Sch_{/S, fppf}$,
$\Vect_1(\mcX/S)(U)$ is the core (maximal subgroupoid) of the $1$--category of
vector bundles over $U \times_S \mcX$

\item For any morphism $q : U \to V \in \Sch_{/S, fppf}$, the functor
\[
\Vect_1(\mcX/S)(V) \to \Vect_1(\mcX/S)(U)
\]
is defined by pullback of vector bundles along
$q \times_S \id_\mcX : U \times_S \mcX \to V \times_S \mcX$.
\end{itemize}
\end{thm}
\begin{proof}
\todo[inline]{Complete the proof}
\end{proof}

\begin{thm}
For any stack $\mcX \in \St_{/S}$ and any scheme $U \in \Sch_{/S}$,
the category $\Vect_1(\mcX/S)$ has object set:
\begin{align*}
 & \Ob{\Vect_1(\mcX/S)} \\
=& \set[{
  (p, \mc{E}_0, \mc{E}_1, f)
}]{
  \parbox{0.65\textwidth}{
    $p : U \to S \in \Ob{(\Sch_{/S})_{fppf}}$, \\
    $\mc{E}_0, \mc{E}_1 \to U \times_S \mcX$ are vector bundles, \\
    $f : \mc{E}_0 \to \mc{E}_1$ is a morphism of vector bundles
  }
}
\end{align*}
and, for two objects $A_i = (p_i, \mc{E}_{0, i}, \mc{E}_{1, i}, f_i)$ where
$i \in \set{0, 1}$, the morphism sets:
\begin{align*}
 & \Hom(A_1, A_2) \\
=& \set[{(q, \alpha_0, \alpha_1)}]{
  \parbox{0.65\textwidth}{
    $q : U_0 \to U_1 \in \Sch_{/S}(p_1, p_2)$, \\
    $\alpha_j$ are vector bundle isomorphisms, such that\\
    \begin{center}
    \begin{tikzcd}[ampersand replacement=\&]
      q^*\mc{E}_{0, 1} \ar[r, "\alpha_1" above] \ar[d, "q^*f_2" left] \&
      \mc{E}_{0, 0} \ar[d, "f_1"] \\
      q^*\mc{E}_{1, 1} \ar[r, "\alpha_2" below] \&
      \mc{E}_{1, 0} \&
    \end{tikzcd}
    \end{center}
    is commutative
  }
}
\end{align*}
\end{thm}
\begin{proof}
The Grothendieck fibration $\Vect_1(\mcX/S) \to \Sch_{/S, fppf}$ is the
Grothendieck construction of the pseudofunctor of \cref{thm:Vect1-psfunctor}.
The result now follows from the concrete description of the Grothendieck
construction.
\end{proof}

\subsection{The Boundary Map}

\begin{defn}
For any stack $\mcX \in \St_{/S}$, taking $s, t$ to be the source and target
maps as defined in \cite[\S 3]{ModQuivBun}, we call the morphism of stacks
$\Vect_1(\mcX/S) \to[{(s, t)}] \Vect(\mcX/S)$, the boundary map.
\end{defn}

\begin{rmk}
The name ``boundary map'' is inspired by the map of simplicial sets
that sends the $1$--simplex to its boundary: that is, the disjoint union
of its source and target vertices.
\end{rmk}

\begin{thm}\label{thm:boundary-map-fibre-functor}
For any stack $\mcX \in \St_{/S}$ and any scheme $U \in \Sch_{/S}$,
the functor $\Vect_1(\mcX/S)(U) \to[{(s, t)}] \Vect(\mcX/S)(U)^2$
is defined by:
\[\begin{array}{ccc}
(\mc{E}_1, \mc{E}_2, f : \mc{E}_1 \to \mc{E}_2)
&\mapsto& (\mc{E}_1, \mc{E}_2) \\
(\alpha_1, \alpha_2) & \mapsto & (\alpha_1, \alpha_2)
\end{array}\]
In particular, the functor is faithful.
\end{thm}
\begin{proof}
\todo[inline]{Complete the proof}
\end{proof}

\begin{cor}
For any stack $\mcX \in \St_{/S}$, if $\Vect_1(\mcX/S)$ and $\Vect(\mcX/S)$
are algebraic, then the morphism of stacks
$\Vect_1(\mcX/S) \to \Vect(\mcX/S)$ is representable by algebraic spaces.
\end{cor}
\begin{proof}
This follows from \cref{thm:boundary-map-fibre-functor} and
\cite[\href{https://stacks.math.columbia.edu/tag/04Y5}{Lemma 04Y5}]
{stacks-project}
\end{proof}

%==============================================================================%

\section{Quiver Bundles on Formal Groupoids}

\subsection{Arrow Bundles on Formal Groupoids}

\begin{thm}\label{thm:Vect1-form-grpd-psfunctor}
Let $(X, \sF)$ be a formal groupoid over $S$ and let
$\Lambda \in \QCoh(X)$ be the corresponding sheaf of rings of differential
operators. Then, the pseudofunctor associated to $\Vect_1(X_\sF/S)$ has the
following concrete description:
\begin{itemize}
\item For each scheme $U \in \Sch_{/S, fppf}$, denoting
the projection $U \times_S X \to X$ by $\pi_U$,
$\Vect_1(X_\sF/S)(U)$ is the groupoid with object set:
\[
\set[{(E_0, \phi_0, E_1, \phi_1, f)}]{
  \parbox{0.5\textwidth}{
  $\mc{E}_i$ are vector bundles on $U \times_S X$, \\
  $\phi_i$ are $\pi_U^*\Lambda$--module structures on $E_i$, \\
  $f : E_0 \to E_1$ is a morphism of $\pi_U^*\Lambda$--modules
  }
}
\]
For each pair of objects
$A_i = (E_{1, i}, \phi_{1, i}, E_{2, i}, \phi_{2, i}, f_i)$, $i \in \set{0, 1}$,
the morphism set $\Hom(A_0, A_1)$ is:
\[
\set[{(\alpha_0, \alpha_1)}]{
  \parbox{0.55\textwidth}{
    $\alpha_j$ are isomorphisms of $\pi_U^*\Lambda$--modules, such that\\
    \begin{center}
    \begin{tikzcd}[ampersand replacement=\&]
      \mc{E}_{0, 0} \ar[r, "\alpha_0" above] \ar[d, "f_2" left] \&
      \mc{E}_{0, 1} \ar[d, "f_1"] \\
      \mc{E}_{1, 0} \ar[r, "\alpha_1" below] \&
      \mc{E}_{1, 1} \&
    \end{tikzcd}
    \end{center}
    is commutative
  }
}
\]

\item For each morphism $q : U \to V$, the functor
\[
\Vect_1(X_\sF/S)(V) \to \Vect_1(X_\sF/S)(U)
\]
is given by pullback along $q \times_S \id_X : U \times_S X \to V \times_S X$
in the following way:
$\pi_U^*F = (\pi_V \circ (q \times_S \id_X))^*F
\simeq (q \times_S \id_X)^*\pi_V^*F$ for all $F \in \QCoh(V \times_S X)$,
so that for any object $(E_0, \phi_0, E_1, \phi_1, f)$ over $V$,
$(q \times_S \id_X)^*\phi_i$ are $\pi_U^*\Lambda$--module structures on
$(q \times_S \id_X)^*E_i$ and $f$ is morphism of $\pi_U^*\Lambda$--modules.
The function of $\Hom$ sets is defined similarly by pullback.
\end{itemize}
\end{thm}
\begin{proof}
\todo[inline]{Complete the proof: should be able use \cref{thm:Vect1-psfunctor}
along with the identification of $\Lambda$--modules with vector bundles on
$X_\sF$ --- for example, as described in \cite[30]{NonAbHodgeFilt}}
\end{proof}

\begin{thm}\label{thm:Vect-alg-Vect1-alg}
For any smooth formal groupoid $(X, \sF)$ over a scheme $S$,
if $\Vect(X/S), \Vect_1(X/S)$ and $\Vect(X_\sF/S)$ are algebraic stacks, then so
is $\Vect_1(X_\sF/S)$.
\end{thm}
\begin{proof}
Using the concrete description of
$\Vect_1(\mcX_\sF/S)$ and $\Vect(X/S)$, viewed as pseudofunctors,
from \cref{thm:Vect1-form-grpd-psfunctor} and \cref{thm:Vect1-psfunctor},
we consider the map of stacks
$F : \Vect_1(\mcX_\sF/S) \to \Vect_1(X_\sF)$
defined for the corresponding pseudofunctors as follows.
For a scheme $U \in \Sch_{/S, fppf}$, the component map --- that is, the
functor of $U$--points --- is given by:
\[\begin{array}{ccccc}
F_U & : & \Vect_1(\mcX_\sF/S)(U) & \to & \Vect_1(X/S)(U) \\
    & : & (E_0, \phi_0, E_1, \phi_1, f) & \mapsto & (E_0, E_1, f) \\
    & : & (\alpha_0, \alpha_1) & \mapsto & (\alpha_0, \alpha_1)
\end{array}\]
recalling that morphisms of $\pi_U^*\Lambda$--modules are, in particular,
morphisms of the underlying vector bundles. Naturality of the $F$
in $U$ is immediate \todo{How?}.

We then compose with $B := (s, t) : \Vect_1(X/S) \to \Vect(X/S)^2$, the boundary
map, to obtain a map $F' : \Vect_1(X_\sF/S) \to \Vect(X/S)^2$. We claim that
$F'$ is representable by algebraic spaces. For this, we will show, for
some arbitrary scheme $U \in \Sch_{/S, fppf}$ and a map
$u = (u_0, u_1) : U \to \Vect(X/S)^2$, that the fibre product
stack/pseudofunctor $U \times_{\Vect(X/S)^2, F', u} \Vect_1(X_\sF/S)$ is
representable by an algebraic space. Let $E_i'$ be the vector bundle
on $U \times_S X$ corresponding to the map $u_i : U \to \Vect(X/S)$.
We then compute the groupoid of
$V$--points for a scheme $V \in \Sch_{/S, fppf}$ as follows:
\begin{itemize}
\item The objects are triples
$(q, E_0, \phi_0, E_1, \phi_1, f, \beta_0, \beta_1)$ where:
  \begin{itemize}[itemsep=0pt]
  \item $q : V \to U$ is a morphism in $Sch_{/S, fppf}$,
  \item $E_i$ are vector bundles on $V \times_S X$,
  \item $\phi_i$ is a $\pi_V^*\Lambda$--module structure on $E_i$,
  \item $f : E_0 \to E_1$ is a morphism of $\pi_V^*\Lambda$--modules,
  \item $\beta_i$ is an isomorphism of vector bundles
  $q^*E_i' \to E_i$ such that the following diagram commutes:
  \[\begin{tikzcd}
  q^*E_0' \ar[r, "\beta_0"] \ar[d, "q^*(f)" left] &
  E_0 \ar[d, "f"] \\
  q^*E_1' \ar[r, "\beta_1" below] & E_1
  \end{tikzcd}\]
  \end{itemize}

\item For objects
$A_i = (q_i, E_{0, i}, \phi_{0, i}, E_{1, i}, \phi_{1, i}, f_i,
\beta_{0, i}, \beta_{1, i})$, $i \in \set{0, 1}$, a morphism
$A_0 \to A_1$ is simply a pair of vector bundle morphisms
$\alpha_{i} : E_{i, }$ ...... such that the following diagram commutes:
\[\begin{tikzcd}
  q_0^*E_0' \ar[rr] \ar[rd] \ar[dd] & &
  E_{0, 0} \ar[dd] \ar[rd] & \\
  & q_1^*E_{0}' \ar[rr, crossing over] & & E_{0, 1} \ar[dd] \\
  q_0^*E_1' \ar[rr] \ar[rd] & & E_{1, 0} \ar[rd] & \\
  & q_1^*E_1' \ar[rr] \ar[from=uu, crossing over] & & E_{1, 1}
\end{tikzcd}\]
\end{itemize}
\end{proof}

\begin{cor}
For any smooth formal groupoid $(X, \sF)$ over an algebraically closed field
$k$ of characteristic zero, where $X$ is a projective variety $k$,
the stack $\Vect_1(X_\sF/S)$ is algebraic.
\end{cor}
\begin{proof}
The stack $\Vect(X_\sF/S)$ is algebraic by \cref{thm:Vect-proj-form-st},
and the result follows from \cref{thm:Vect-alg-Vect1-alg}.
\end{proof}

\printbibliography

\end{document}
