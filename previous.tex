We will work with fppf stacks
(which includes schemes, algebraic spaces, algebraic stacks,
formal algebraic spaces, formal algebraic stacks)
over a base scheme $S$. We denote the category of such objects $\St_{/S}$.
We wish to prove the algebraicity of a stack whose $k$--points are morphisms of
connections. By \cites{NonAbHodgeFilt, GeomNonAbHodgeFilt, ModQuivBun},
this stack should be the pullback:
\[\begin{tikzcd}
\mc{M}_{\Vect(X_{DR}), 1}
  \ar[r] \ar[d] \ar[rd, "\lrcorner" very near start, phantom] &
\Map(X_{DR}, \mc{E}_{X_\Lambda}^\vee \boxtimes \mc{E}_{X_\Lambda})) \ar[d] \\
\mathrm{pt} \ar[r, "\overline{\Delta}"] &
\Map(X_{DR}, X_{DR} \times X_{DR})
\end{tikzcd}\]
To prove the algebraicity of the pullback, it suffices to show that the
right and bottom left vertices are algebraic, and that the bottom right
vertex has a diagonal representable by algebraic spaces
\cite[\href{https://stacks.math.columbia.edu/tag/04TF}{Lemma 04TF}]
{stacks-project}.

We note that $X_{DR}$ is an example of a formal groupoid defined in
\cite[23]{AlgGeom-n-St}, and we will try to show the result for
an arbitrary formal groupoid $X_\Lambda$. By varying ``$\Lambda$'' in a sense
that we can make precise, this simultaneously
shows the algebraicity of the moduli stacks of morphisms of Higgs bundles,
morphisms of $\lambda$--connections for each $\lambda \in k$, and the moduli
stack fibred over $\bb{A}^1_k$ and parametrizing morphisms of all
$\lambda$--connections
for all $\lambda \in k$. The last example gives us a way to address non-Abelian
Hodge theory from the stack theoretic perspective ---
namely, the ``preferred sections'' approach of Deligne and Simpson
\cites{NonAbHodgeFilt,GeomNonAbHodgeFilt}.

\section{Strategy (algebraicity of
{$\Map(X_\Lambda, \mc{E}_{X_\Lambda}^\vee \boxtimes \mc{E}_{X_\Lambda})$}):
Show that the target is very presentable \cite[69]{Prsntbl-n-St}}

The goal is to apply something like \cite[Lemma 4.23]{Prsntbl-n-St} and
\cite[Theorem 7.2]{AlgGeom-n-St}. The paragraph before
\cite[Proposition 5.1]{AlgGeom-n-St} has a simple characterization of very
presentability for a connected $n$--stack $T$ over an algebraically closed field
$k$: $T$ is very presentable if and only if for each point $t \in T(\Spec(k))$,
$\pi_i(T, t)$ are vector spaces and each $i \geq 2$, and $\pi_1(T, t)$ is an
affine group scheme. For $n = 1$, this reduces to $\pi_1(T, t)$ being an
affine group scheme. For this, we will try to show the following:

\begin{thm}
For an affine morphism of stacks (over a field $k$) $p : \mc{E} \to Y_\Lambda$
where:
\begin{enumerate}
  \item $Y_\Lambda$ is the stack associated to a formal groupoid $(Y, \Lambda)$,
  \item $Y_\Lambda$ has affine stabilizers of $k$--points, and
  \item $\mc{E}$ is connected,
\end{enumerate}
$\mc{E}$ is very presentable.
\end{thm}
\begin{proof}
Since $\mc{E}$ is a connected $1$--stack, we need only check that for the
ground field $k$ and all $e : \Spec(k) \to \mc{E}$, $\pi_1(\mc{E}, e)$ is an
affine scheme by \cite[paragraph before Proposition 5.1]{AlgGeom-n-St}. The
sheaf $\pi_1(\mc{E}, e)$ is $\pi_0(\Stab_\mc{E}(e))$ by
\cref{lem:stabilizer-to-pi1}.
However, since $p$ is affine and $Y_\Lambda$ has affine stabilizers, $\mc{E}$
has affine stabilizers.\footnote{This is well known
--- see, for example, \cite[Lemma 4.9]{ModQuivBun}.}
Thus, $\Stab_\mc{E}(e)$ is a sheaf of sets, in particular, and we have
$\pi_0(\Stab_\mc{E}(e)) = \Stab_\mc{E}(e)$. This shows that $\pi_1(\mc{E}, e)$
is affine.
\end{proof}

\begin{lem}\label{lem:stabilizer-to-pi1}
Given any $1$--stack $T$ and a morphism $t : S \to T$ from a scheme,
$\pi_1(T, t) = \pi_0(\Stab_T(t))$.
\end{lem}
\begin{proof}
\todo[inline]{Complete this.}
\end{proof}

\begin{thm}
Let $S$ be a scheme over an algebraically closed field $k$, and
$(X, \Lambda)$, a formal groupoid in $\Sh_{\et}(\CRing^{\op})_{/S}$ with
associated stack $X_\Lambda \to S$. Then, the stabilizers of points
$x : \Spec(k) \to X_\Lambda$ are subschemes of $S$.
\end{thm}
\begin{proof}
Let $x : \Spec(k) \to X_{\Lambda}$ be a morphism for some arbitrary field $k$.
By \cref{lem:frml-grpd-frml-st}, the map $X \to X_{Hod}$ is surjection,
providing a lift $x' : \Spec(k) \to X$ of $x$ by \cref{lem:lift-point-atlas}.
These fit into a gluing of Cartesian squares:
\[\begin{tikzcd}
\Stab_{X_\Lambda}(x) \ar[r] \ar[d] &
s^{-1}(x') \ar[r] \ar[d] &
\Spec(k) \ar[d, "x'"] \\
t^{-1}(x') \ar[r] \ar[d] &
\Lambda \ar[r, "s"] \ar[d, "t"] &
X \ar[d] \\
\Spec(k) \ar[r, "x'" below] &
X \ar[r] &
X_\Lambda
\end{tikzcd}\]
We observe $x'$ is a monomorphism so that all maps except the ones in the bottom
right square are monomorphisms by stability of monomorphisms under pullback. We
also observe that $x'$ is of finite type since $X$ is of finite type by
assumption.
\end{proof}

\begin{lem}\label{lem:lift-point-atlas}
Given an atlas $U \to F$ for a formal algebraic stack $F$, every map
$x : S \to F$ from a scheme $S$, has a $2$--factorization
$S \to[x'] U \to F$.
\end{lem}
\begin{proof}
\todo[inline]{Complete this.}
\end{proof}

\begin{lem}\label{lem:frml-grpd-frml-st}
The quotient stack $X_\Lambda$ of a formal groupoid $(X, \Lambda)$ of smooth
type\footnote{This is a groupoid object in formal schemes where
$X$ is the object of objects,
$\Lambda$ is the object of morphisms,
$X$ is a smooth scheme,
$\Lambda$ is a formally smooth formal scheme,
all structure morphisms are smooth,
and the identity morphism $X \to \Lambda$ realizes $X$ as the underlying scheme
of $\Lambda$. See \cite[29]{NonAbHodgeFilt} for a definition.}
is a formal algebraic stack, as defined in \cite[Definition 5.3]{FormalAlgSt},
with an atlas given by the canonical morphism $X \to X_\Lambda$.
\end{lem}
\begin{proof}
\todo[inline]{Complete this.}
\end{proof}

\begin{cor}
The quotient stack $X_\Lambda$ of a formal groupoid $(X, \Lambda)$ of smooth
type has a diagonal representable by algebraic spaces and locally of finite
type.
\end{cor}
\begin{proof}
Combine \cref{lem:frml-grpd-frml-st} with \cite[5.12]{FormalAlgSt}.
\end{proof}

\begin{thm}
For a projective variety $X$ over an algebraically closed field $k$,
the stack
\[
\Map_{\bb{A}^1_k}(X_{Hod}, \Vect) : (Z \to \bb{A}^1_k)
\mapsto \St(X_{Hod} \times_{\bb{A}^1_k} Z, \Vect)
\]
has affine stabilizers.
\end{thm}

\section{Strategy (algebraicity of
{$\Map(X_\Lambda, \mc{E}_{X_\Lambda}^\vee \boxtimes \mc{E}_{X_\Lambda})$)}:
Verify Artin's axioms}

\section{Strategy (representability of the diagonal of {$X_\Lambda$}):
First show that ($X_\Lambda$) has a representable diagonal}

Adapt \cite[Theorem 3.4.13]{AlperModuli} to prove that quotients of
smooth formal groupoids, as defined in say \cite[23]{AlgGeom-n-St},
are formal algebraic stacks, as defined in \cite[Definition 5.3.]{FormalAlgSt}.
Then, \cite[Lemma 5.12]{FormalAlgSt} shows that $X_\Lambda$ has a diagonal
representable by algebraic spaces and locally of finite type. Use this
show the representability of the diagonal of
$\Map(X_\Lambda, X_\Lambda \times X_\Lambda)$.

\section{Strategy (Direct): Representable map of section stacks}

Consider the pullback squares:
\[\begin{tikzcd}
\mc{F}
  \ar[r] \ar[d]
  \ar[rd, "\lrcorner" very near start, phantom] &
\mc{E}_{X_\Lambda}^\vee \boxtimes \mc{E}_{X_\Lambda}
  \ar[d] \\
X_\Lambda'
  \ar[r] \ar[d] \ar[rd, "\lrcorner" very near start, phantom] &
(\mc{M}_{\Perf(X_\Lambda)} \times X_\Lambda)^2
  \ar[d] \\
X_\Lambda \ar[r, "\Delta_{X_\Lambda}" below] &
X_\Lambda \times X_\Lambda
\end{tikzcd}\]

Then, we have
\[
\mc{M}_{\Perf(X_\Lambda), 1} = \Gamma(X_\Lambda, \mc{F})
\]
Find a map:
\[
\Gamma(X_\Lambda, \mc{F}) \to \Gamma(X_\Lambda', \mc{F})
\text{ or } \Gamma(X_\Lambda', \mc{F}) \to \Gamma(X_\Lambda, \mc{F})
\]
given by composing with the pullback projection $X_\Lambda' \to X_\Lambda$,
using the universal property of pullbacks. The stack $\Gamma(X_\Lambda', \mc{F})$
is geometric $n$-stack by \cite[Corollary 6.4]{AlgGeom-n-St}. If we can show
that the first above morphism is geometric, then
$\mc{M}_{\Perf(X_\Lambda), 1} = \Gamma(X_\Lambda, \mc{F})$ should be geometric
by \cite[Corollary 2.6]{AlgGeom-n-St}. If we can show that the second above
morphism is geometric, surjective and smooth, then
$\mc{M}_{\Perf(X_\Lambda), 1} = \Gamma(X_\Lambda, \mc{F})$
is geometric by \cite[Lemma 2.4]{AlgGeom-n-St}.

